\documentclass[notes]{beamer}
%
% Choose how your presentation looks.
%
% For more themes, color themes and font themes, see:
% http://deic.uab.es/~iblanes/beamer_gallery/index_by_theme.html
%
\mode<presentation>
{
  \usetheme{Madrid}      % or try Darmstadt, Madrid, Warsaw, ...
  \usecolortheme{default} % or try albatross, beaver, crane, ...
  \usefonttheme{default}  % or try serif, structurebold, ...
  \setbeamertemplate{navigation symbols}{}
  \setbeamertemplate{caption}[numbered]
} 

\usepackage[english]{babel}
\usepackage[utf8]{inputenc}
\usepackage[T1]{fontenc}
\usepackage{import}
%\usepackage{enumitem}

\usepackage[
backend=biber,
style=numeric,
sorting=none
]{biblatex}

\usepackage{csquotes}

\bibliography{references_list.bib}
% Removes icon in bibliography
\setbeamertemplate{bibliography item}{\insertbiblabel}



\title[Reciprocal Concept Map Collaboration]{
%Reciprocal kit building
%for collaboration: An analysis and evaluation of
%collaborative task in the practical classroom
% Analysis and evaluation of Reciprocal Kit Building
% for collaborative problem solving in a practical classroom
Reciprocal Concept Map Collaboration: Evaluating Group Products of Re-constructional Concept Maps
}

% The effect of RKB application 
% To what extent the RKB approach influence collaborative product


\author{Lia Sadita}
\institute{Hiroshima University}
\date{May, 2020}

\AtBeginSection[]
{
    \begin{frame}[allowframebreaks]% <beamer>
    \frametitle{Outline for section \thesection}
    \tableofcontents[currentsection,currentsubsection]
    \end{frame}
}

\begin{document}

\begin{frame}
  \titlepage
\end{frame}

% Uncomment these lines for an automatically generated outline.
% \begin{frame}{Outline}
%   \tableofcontents 
% \end{frame}


\section{Introduction}
\import{./}{intro.tex}

\section{Methods}
\import{./}{method.tex}

\section{The effectiveness of the RKB approach for group collaboration}
\import{./}{a1_firstanalysis.tex}

\section{The effect of differences in group composition on 
knowledge transfer, group achievement and learners' affective}
\import{./}{a2_secondanalysis.tex}

\section{Analysis of the similarity of 
individual knowledge and the comprehension of partner's representation during 
collaborative concept mapping}
\import{./}{a3_thirdanalysis.tex}

\section{Conclusion}
\import{./}{conclusion.tex}

\begin{frame}[allowframebreaks]
\frametitle{References}

% This prints the bibliography on the slide
\printbibliography
\end{frame}

\end{document}
