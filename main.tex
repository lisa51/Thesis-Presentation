\documentclass[notes]{beamer}
%\documentclass[11pt,handout]{beamer} % HANDOUT print
%
% Choose how your presentation looks.
%
% For more themes, color themes and font themes, see:
% http://deic.uab.es/~iblanes/beamer_gallery/index_by_theme.html
%
\mode<presentation>
{
  \usetheme{Antibes}      % or try Darmstadt, Madrid, Warsaw, ...
  \usecolortheme{default} % or try albatross, beaver, crane, ...
  \usefonttheme{default}  % or try serif, structurebold, ...
  \setbeamertemplate{navigation symbols}{}
  \setbeamertemplate{caption}[numbered]
} 

\usepackage[english]{babel}
\usepackage[utf8]{inputenc}
\usepackage[T1]{fontenc}
\usepackage{import}
%\usepackage{xcolor}
\usepackage{CJKutf8}
\usepackage{xcolor}

\usepackage[
backend=biber,
style=numeric,
sorting=none
]{biblatex}

\usepackage{csquotes}

\bibliography{references_list.bib}
% Removes icon in bibliography
\setbeamertemplate{bibliography item}{\insertbiblabel}

%Reciprocal kit building
%for collaboration: An analysis and evaluation of
%collaborative task in the practical classroom
% Analysis and evaluation of Reciprocal Kit Building
% for collaborative problem solving in a practical classroom
% Reciprocal Concept Map Collaboration: Evaluating Group Products of Re-constructional Concept Maps
% Collaboration with Reciprocal Kit-Build


% \emph{PhD Preliminary Defense}
\title[Collaboration with Reciprocal KB]{
    Collaboration with Reciprocal Kit-Build: An analysis of group products
}

% \begin{CJK}{UTF8}{min}
\subtitle{(相互キットビルドコンセプトマップを用いたコラボレーション:グループ生産物の分析)}
% \end{CJK}
\author[Lia S.]{Lia Sadita (D171171)}
\institute[Hiroshima Univ.]{
    Learning Engineering Laboratory,
    Dept. of Information Engineering \\ 
    Hiroshima University}
\date{June, 2020}

\addtobeamertemplate{title page}{\begin{center}{\small PhD Preliminary Defense}\end{center}}{
\footnotesize{\begin{center}
Supervisors: 
Prof. Tsukasa Hirashima, Prof. Kazufumi Kaneda, \& Dr. Yusuke Hayashi
\end{center}}}

% Supervisors:
% Prof. Tsukasa Hirashima
% Prof. Kazufumi Kaneda
% Dr. Yusuke Hayashi

\logo{\includegraphics[height=1cm]{images/logo-hu.png}}

\AtBeginSection[]
{

    \ifnum \value{framenumber}>4
      \begin{frame}<beamer>
      \frametitle{Outline for section \thesection}
      \tableofcontents[sections=\value{section}]
      \end{frame}
    \else
    \fi
    
}

\begin{document}
\begin{CJK}{UTF8}{min}
\begin{frame}
  \titlepage
\end{frame}

%\section{Summary}
\import{./}{summary.tex}

% Uncomment these lines for an automatically generated outline.
\begin{frame}[allowframebreaks]{Outline}
  \tableofcontents[hideallsubsections] 
\end{frame}

\section{Introduction}
\import{./}{intro.tex}

\section{Methods}
\import{./}{method.tex}

\section{RKB evaluation on collaborative products and students’perspective toward the activities}
\import{./}{a1_groupcollab.tex}

\section{The effect of individual knowledge differences to group collaboration}
\import{./}{a2_knowledgediff.tex}

\section{Predicting collaborative products based on
similarity of knowledge and comprehension of partner's 
representation}
\import{./}{a3_prediction.tex}

\section{Conclusion}
\import{./}{conclusion.tex}

\begin{frame}[allowframebreaks]
\frametitle{References}
% This prints the bibliography on the slide
\printbibliography
\end{frame}

\begin{frame}{Thank you}
    Thank your for your attention
\end{frame}

\section{Supplementary}
\import{./}{supplementary.tex}

\end{CJK}
\end{document}
