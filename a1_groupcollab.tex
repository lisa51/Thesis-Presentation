\subsection{Motivation}
\begin{frame}{Research objectives}

\begin{block}{Aim}
    The focus of this particular study is to identify the effectiveness
    of approach for creating a collaborative product.
    % Does the RKB approach affect co-construction of knowledge with concept maps? 
    %If so, to what extent?
\end{block}

\begin{alertblock}{Sub-research questions}
    \begin{enumerate}
        % \item What is the score of group maps? % descriptive RQs
        % \item What is the score of individual maps?
        \item Is there a significant different between the learning outcomes as a group
        or as an individual?
        \item How is the pattern of map changes from individual to group?
        \item How is the affective response of the students while following the activities?
\end{enumerate}
\end{alertblock}

%% The study focuses on identifying the effectiveness from the end-products 
%% (collaborative maps), the patterns of map changes from individuals to groups, 
%% and the perceptions of the students while following the learning activities. 
\end{frame}

\subsection{Analysis methods}

\begin{frame}{Methods}
    \begin{alertblock}{Sub-research questions}
        \begin{enumerate}
        \item Is there a significant different between the learning outcomes as a group
        or as an individual?
        \begin{itemize}
            \item evaluation of individual and group concept map by the teacher 
                  as a domain expert
        \end{itemize}
        
        \item How is the pattern of map changes from individual to group?
        \begin{itemize}
            \item a propositional-level similarity analysis
        \end{itemize}
        
        \item How is the affective response of the students while following the activities?
        \begin{itemize}
            \item conduct a survey on learners' experiences during each phase of RKB activities
        \end{itemize}
        \end{enumerate}
    \end{alertblock}
\end{frame}

\begin{frame}[allowframebreaks]{Questionnaire on learner's affective response}
    \begin{itemize}
        \item The survey covered items regarding the perspectives of 
              the students on the task itself (e.g., attractiveness 
              and stimulation scales) and the system used (or non-task; 
              e.g., perspicuity scale). 
        \item The questionnaire scales were adopted from the User Experience
              Questionnaire, an Indonesian version (Laugwitz, Held, \& Schrepp, 2008;
              Santoso, Schrepp, Kartono, Yudha, \& Priyogi, 2016). 
        \item The six open-ended questions were given to uncover the positive and
              negative experiences of the students during the experiment.
    \end{itemize}
    
    \begin{figure}[tb]
        \begin{center}
            \includegraphics[width=100mm]{images/rqa_questionnaire_closed_items.pdf}
        \end{center}
        \caption{Closed-ended items}
        \label{questionnaire_close}
    \end{figure}
    
    \begin{figure}[tb]
        \begin{center}
            \includegraphics[width=100mm]{images/rqa_questionnaire_open_items.pdf}
        \end{center}
        \caption{Open-ended items}
        \label{questionnaire_open}
    \end{figure}
\end{frame}

\subsection{Results \& discussions}
\begin{frame}[allowframebreaks]{Results (1): Overall group performance}
\begin{table}[tb]
    \caption{Descriptive statistics}
    \label{a1::group_performance}
    \begin{center}
        \begin{tabular}{c|c|c}
            \hline
            & Individual-map score & Group-map score\\
            \hline
            $M$ & 72.21 & 90 \\
            $SD$ & 18.22 & 7.31 \\
            $Min.$ & 41.43 & 75.71 \\
            $Max.$ & 98.57 & 100 \\
            \hline
        \end{tabular}
    \end{center}
\end{table}

\begin{figure}[tb]
    \begin{center}
        \includegraphics[width=80mm]{images/a1_mapscore_distribution.pdf}
    \end{center}
    \caption{Scores from the average of individual and group maps, along with the differences between the two scores}
    \label{a1::mapscore_distribution}
\end{figure}

\begin{table}[tb]
    \caption{Distribution of correctness level in all individual and group propositions}
    \label{dist_correct}
    \begin{center}
        \begin{tabular}{ p{6cm}|p{1.5cm}|p{1.5cm}  }
            \hline
            Level of correctness & Individual-map (\%) & Group-map (\%)\\
            \hline
            The true proposition & 64 & 81 \\
            The false proposition with a minor error & 5 & 7 \\
            The false proposition with a moderate error & 10 & 7 \\
            The false proposition with a fatal error & 21 & 5 \\
            \hline
        \end{tabular}
    \end{center}
\end{table}

\begin{block}{Finding \#1}
    Almost all groups produce high-quality collaborative products.
\end{block}

\end{frame}



\begin{frame}{Questionnaire results}
    \begin{figure}[tb]
    \begin{center}
        \includegraphics[width=100mm]{images/rqa_affective_response.pdf}
    \end{center}
    \caption{Sample of individual maps and group map generated by group ALG12}
    \label{a1::questionnaire}
\end{figure}

\begin{block}{Finding \#2}
    Students show positive acceptance toward the activities. 
\end{block}

\end{frame}

\begin{frame}{Pattern of map changes}
    \begin{figure}[tb]
    \begin{center}
        \includegraphics[width=80mm]{images/rqa_map_patterns_a.pdf}
    \end{center}
    \caption{Proportions of the individual propositions taken from the system,with  the  matching,  lacking,  and  excessive  links,  compared  to  the  group propositions}
    \label{a1::map_sample_1}
\end{figure}
\end{frame}


\begin{frame}{Pattern of map changes (2)}
    \begin{figure}[tb]
    \begin{center}
        \includegraphics[width=80mm]{images/rqa_map_patterns_b.pdf}
    \end{center}
    \caption{The  number  of  propositions  from  the  individual  to  the  groupmaps were categorized by the link types in the KB-map, the similarity be-tween the KB-map proposition and the group-map proposition, and the levelof group proposition correctness}
    \label{a1::map_sample_2}
\end{figure}
\begin{block}{Finding \#3}
    High correlation between the visualization of map differences
    and score changes, modification were made
    from the excessive and/or lacking links. 
\end{block}

\end{frame}
