\subsection{Summary of findings}
\begin{frame}{Main findings}
    \begin{enumerate}
    \item <1-> The students build \textcolor{teal}{high-quality group products}. 
    \item <2-> The students perceived \textcolor{teal}{positive responses} toward the activities. 
    \item <3-> Group formation based on the similarity of individual 
    knowledge \textcolor{teal}{does not significantly affect the individual-to-group 
    transfer of knowledge and the group outcomes}, however, 
    it may induce students to \textcolor{purple}{experience different affective states}.
    \item <4-> \textcolor{teal}{The comprehension of partner's representation is a 
    stronger predictor} to estimate the group outcomes compare to the similarity of initial knowledge. 
\end{enumerate}  
\end{frame}

\subsection{Study limitations and directions for future works}
\begin{frame}{Limitations of the current study}
    \begin{itemize}
        \item <1>\textcolor{purple}{Generalizability of findings}\\
        {\small The concept mapping activity was conducted once during two hours 
        of a class session. It is may insufficient to infer the generalizability of
        the results, more experimental session during one term of 
        study is strongly recommended.}
        \item <2>\textcolor{purple}{Single group study design}\\
        {\small A single group study was conducted to ensure fairness in a real classroom context. It would also be interesting in the future to compare the results of groups with reciprocal teaching activity and conventional collaborative concept mapping}
    \end{itemize}
\end{frame}

\begin{frame}{Limitations of the current study (cont'd)}
    \begin{itemize}
        \item \textcolor{purple}{The evaluation of learning effectiveness at a group and interaction level}\\
        {\small We excluded consideration of the 
        effectiveness at the level of the individual since 
        we did not collect individual post-collaboration maps, due to
        time limitations enforced by conducting the experiment in a practical classroom.
        An assessment of learning effect at the individual level is 
        necessary to comprehend the current findings. }
    \end{itemize}
\end{frame}

\begin{frame}{Another potential future works}
    \begin{itemize}
        \item Development of a supporting function for collaboration based on the reconstructed maps\\
        \item Bigger group size (more than 2 people)\\
    \end{itemize}
\end{frame}

\begin{frame}{}
    Thank you for your attention :-)
\end{frame}

